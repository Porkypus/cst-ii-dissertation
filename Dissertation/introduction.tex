% The Introduction should explain the principal motivation for the project. Show how the work fits into the broad 
% area of surrounding Computer Science and give a brief survey of previous related work. It should generally be 
% unnecessary to quote at length from technical papers or textbooks. If a simple bibliographic reference is 
% insufficient, consign any lengthy quotation to an appendix.

% ~500 words

\documentclass[final,dissertation.tex]{subfiles}
\begin{document}

\chapter{Introduction}

\section{Motivation}

The motivation behind this project is to improve communication accessibility for individuals who use British Sign Language (BSL). BSL is a visual language that is used by around 150,000 people in the UK, many of whom are deaf or hard of hearing. Despite its prevalence, BSL is not widely understood or recognized by the general population, making communication with individuals who use BSL difficult and often frustrating. Additionally, the project aims to raise awareness and understanding of BSL, and to demonstrate the power of technology to promote accessibility and inclusivity.

\section{Related Work}

Sign language recognition can be grouped into two distinct parts: isolated sign language and continuous sign language. While isolated sign language involves distinct gestures being performed one at a time, continuous signing involves many gestures performed in quick succession, often merging some signs together.

Work has been conducted on the BOBSL\cite{albanie2021bbc} dataset, which is the largest continuous BSL dataset. This work aimed at improving real-time translation of BSL, which is usually performed with continuous signing.

Although this project focuses primarily on BSL, it is still important to mention other work on other variations of sign language due to the similarities in the main task. The MS-ASL dataset is one of the largest datasets for isolated sign language recognition. The study which yielded it also proposed I3D as a novel way of classifying sign gestures.

\end{document}