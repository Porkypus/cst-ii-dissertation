\documentclass[final,rdr32.tex]{subfiles}
\begin{document}

\pagestyle{plain}

\chapter*{Proforma}

 {\large
  \begin{tabular}{ll}
      Name:               & \bf Ronan Ragavoodoo                             \\
      College:            & \bf Homerton College                             \\
      Project Title:      & \bf Sign Language Recognition                    \\
      Examination:        & \bf Computer Science Tripos -- Part II, May 2023 \\
      Word Count:         & \bf 10205\footnotemark[1]                        \\
      Code Line Count:    & \bf 1000\footnotemark[2]                         \\
      Project Originator: & The Author                                       \\
      Supervisor:         & Mr. Filip Svoboda                                \\
  \end{tabular}
 }

\footnotetext[1]{This word count was computed using \VerbatimFootnotes{texcount -inc rdr32.tex}}
\footnotetext[2]{This line count was computed using Visual Studio Code's in-built line count}

\section*{Original Aims of the Project}

% Goal of project, background

The original aim of this project was to create a model that can help recognize British Sign Language (BSL), specifically isolated signs. This model would be able to accurately interpret and translate BSL gestures and movements into written or spoken language, making communication with individuals who use BSL more seamless and accessible. The goal was to develop a program that can assist individuals who are deaf or hard of hearing, as well as improve overall accessibility in society. The project would be focused on utilizing machine learning and computer vision techniques to train the model to recognize BSL gestures with high accuracy.

\section*{Work Completed}

% Sell it better, more numbers

I have successfully implemented a model that can recognise with high-accuracy a variety of isolated BSL expressions in real-time. Different models have been evaluated to compare performances and determine which is more adequate for real-time recognition.

\section*{Special Difficulties}

None.

\end{document}